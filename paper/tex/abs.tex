\begin{abstract}
Covert communication is referred to as a form of a communication channel between two parties of Alice and Bob who want to transfer their messages secretly by hiding the presence of their transmissions from a watchful warden Willie. Many works have strived to address the covert channel problem, however, preserving both the signals' statistical properties and the normal communication performance has scarcely been investigated. In this work, we propose a novel covert communication scheme that can be added to any normal autoencoder-based communication without impacting the performance or the statistics of the transmitted signals. We define the covert problem as an optimization problem wherein three covert actors of Alice, Bob, and Willie are represented by a generator, a decoder, and a discriminator neural network. They are jointly trained in an adversarial setting to establish a covert channel in the form of covert noise signals that have the same statistical properties as of the channel's noise regardless of the channel type. It is also ensured that our added covert noise signals have the lowest impact on the existing normal communication's error rate. Our results show that our learning-based covert scheme is able to successfully establish a reliable undetectable covert channel between Alice and Bob independent of the channel model and cover signals and causes almost no disturbance to the ongoing normal communication of the system.
\end{abstract}

\begin{comment}
	Covert communication, wireless communication, autoencoder wireless systems
\end{comment}