\begin{abstract}
Covert communication is referred to as a form of communication channel between two parties of Alice and Bob who want to transfer their messages secretly by hiding the presence of their transmissions from a watchful warden Willie. Although there exists a huge body of work on covert communication examining multiple aspects of this topic, there is a lack of studies investigating viability of having such channels on recently introduced autoencoder-based wireless systems. In this work, we propose a novel deep learning based covert communication scheme that runs on top of an autoencoder communication system. We define the covert problem as an optimization problem wherein three covert actors of Alice, Bob, and Willie are represented by a generator, a decoder, and a discriminator neural network jointly trained in an adversarial setting. The objective is to establish a covert channel in a form of covert noise signals that have the same statistical properties as of the channel's noise. Additionally, we ensure that our added covert noise signals have the lowest impact on the existing normal communication's error rate. Our results show that our learning-based covert scheme is successfully able to establish a reliable undetectable channel between Alice and Bob while causing almost no disturbance on the ongoing communication of the system.
\end{abstract}

\begin{comment}
	Covert communication, wireless communication, autoencoder wireless systems
\end{comment}