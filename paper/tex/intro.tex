\section{Introduction}
\label{s:intro}
Due to the shared and broadcast nature of wireless channels, there is considerable attention on the security and privacy aspects of wireless communications. While traditional cryptography methods and physical-layer securities can protect the confidentiality of the content (i.e the information transmitted over the channel), there are occasions that hiding the very existence of the communication channel is more vital than securing the communicated message itself. Examples of such situations are military operations, cyber-espionage, social unrest, or communication parties' privacy. All above have motivated the study of hidden communication channels, namely, "covert channels" \cite{lampson1973note}.


The preliminary attempt to obtain covertness started with the study of spread spectrum beginning almost a century ago with the main purpose of hiding military communications. The idea was to spread the transmit power into the noise so that the transmissions are mixed within the noise. Many works continued to further examine different aspects of this idea, however, the fundamental performance limits of such work were unknown until recently when Bash et al. \cite{bash2012square} established a square root limit on the number of covert bits that can be reliably sent over an additive white Gaussian noise (AWGN) channel while staying covert to a channel's observer. Followed by this work, there has been a surge of interest in examining covert channels \cite{sobers2017covert,soltani2018covert} especially in point-to-point wireless communication models.


Numerous works have studied the theoretical limits of covert communication over wireless channels in different scenarios \cite{bash2012square, soltani2018covert, sheikholeslami2018multi, li2021fundamental}, but only few works have focused on a practical implementation of such channels. In a typical covert communication scenario, there is a specific element of the communication system that Alice and Bob leverage to build their covert communication upon. Some of which to mention are: hardware impairments \cite{mohammed2021adversarial}, channel's noise effect \cite{soltani2018covert}, presence of a cooperative jammer \cite{sobers2017covert}. Beyond that, the majority of works are based on favorable assumptions to covert users, such as existing a shared secret key between Alice and Bob unknown to Willie, accessibility of covert users to cover signals and modulation type, uncertainty in the knowledge of noise power at the Willie's receiver, neglecting the impact of covert system on the normal communication, and limiting the channel model to AWGN; all of which have led to impracticability of such methods in real world scenarios. Although there is no argue that these works have successfully demonstrated a novel and an unique method of covert communication, but it is always encouraged to study methods that are free as much as viable of these assumptions.


Furthermore, it is important to study applicability of covert channels on next generation communication systems. Similar to many other areas, machine learning (ML) has now found it way to many wireless communication domains \cite{wang2017deep}. In fact, various network optimization problems, which were traditionally used to be handled with statistical models, are now leveraging on machine learning techniques. Deep neural networks (DNNs) in particular, a major force in machine learning, have answered several wireless problems, such as signal classification, channel estimation, transmitter identification, jamming and anti-jamming \cite{bahramali2021robust}. Very recently, as a replacement for conventional modular-based designs, an end-to-end communication model based on deep learning methods has emerged \cite{o2017introduction}. In this new paradigm, the transmitter and receiver are jointly trained as the encoder and decoder of an autoencoder network. One noticeable difference between end-to-end systems and conventional modular designs is that in end-to-end systems, the encoder and decoder learn the coding and the modulation tasks simultaneously as opposed to having separate modules for each.


In this work, we introduce a novel deep learning based covert communication method that can be augmented to any autoencoder-based wireless communication and establish a reliable covert channel on top of the existing normal communication. Our proposed covert channel works independent of the channel model, requires no knowledge of cover signals or the modulation type, and it is optimized to have the minimum impact on the communication between normal users while being undetectable to a watchful channel's observer. It is worthwhile to mention that even though we are proposing our method for autoencoder-based wireless communication systems, there is no limitation on integrating our model into existing conventional wireless communication systems.


The contributions of this work can be expressed as:
\begin{enumerate}
	\item \textbf{Cover-Agnostic}: We propose a novel covert communication method using GANs that is independent of cover signals, waveforms, and modulation type of a wireless system.
	\item \textbf{Channel-Independent}: In our proposed covert scheme, covert users do not have any information about the channel model nor the noise level.
	\item \textbf{Undetectability}: By forming an adversarial competition between observer and covert users wherein one competes to outperform the other, we ensure the undetectability of covert signals to a high extent.
	\item \textbf{Low Interference}: We propose a covert communication model that can be integerated into any normal communication system while having almost no disturbance on the ongoing communication between normal parties.
\end{enumerate}
