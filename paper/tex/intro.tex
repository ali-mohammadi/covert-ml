\section{Introduction}
\label{s:intro}

Due to the shared and broadcast nature of wireless channels, there is considerable attention on the security and privacy aspects of wireless communications. While traditional cryptography methods and physical-layer securities can protect the confidentiality of the content (i.e the information transmitted over the channel), there are occasions that hiding the very existence of the communication channel is more vital than securing the communicated message itself. Examples of such situations are military operations, cyber-espionage, social unrest, or communication parties' privacy. All above have motivated the study of hidden communication channels, namely, "covert channels" \cite{lampson1973note}.\\
The preliminary attempt to obtain covertness started with the study of spread spectrum, which began almost a century ago with the main purpose of hiding military communications. The idea is to spread the transmit power into the noise so that the communication stays hidden. Many works continued to further examine different aspects of this idea, yet the fundamental performance limits of such work were unknown until recently when Bash et al. \cite{bash2012square} established a square root limit on the number of covert bits that can be reliably sent over an additive white Gaussian noise (AWGN) channel while remaining covert to a channel's observer. Followed by this work, there has been a surge of interest in examining covert channels \cite{sobers2017covert,soltani2018covert} especially in point-to-point wireless communication models.\\
In the last decade, majority of the research works on covert communication focus on integration of covert communication techniques into the current and future wireless technologies, such as 6G wireless networks and IoT. With the advent of AI and recent advances in computer's computation power and algorithmic designs, machine learning (ML) is now becoming an integral part of many research topics. Correspondingly, wireless community started to adopt machine learning techniques and use them as solutions to the various network optimization problems, which were traditionally handled with statistical models. Deep neural networks (DNNs) in particular, a major force in machine learning, are now applied into many of these tasks including signal classification, channel estimation, transmitter identification, jamming and anti-jamming [?].\\
More recently, as a replacement for conventional modular-based designs, an end-to-end communication model based on deep learning methods has emerged \cite{o2017introduction}. In this new paradigm, the transmitter and receiver are jointly trained as the encoder and decoder of an autoencoder network. While the transmitter learns to encode the transmitted symbol to an embedding vector of wireless signals, the decoder, on the other side, learns to retrieve that symbol back from the transmitted signal after passing through the channel. One noticeable difference between end-to-end systems and conventional modular designs is that in end-to-end systems, the encoder learns coding and modulation tasks simultaneously as opposed to having separate modules to perform each task. This is also true for the decoder of the system that jointly learns demodulation and decoding. Given the fact that channel distortions usually have a non-linear behavior, a DNN-based communication model is expected to capture these dissimilitudes better compared to a statistical model. That means, these systems can learn more complex channel effects and can operate more robustly under different noise levels and channel imperfections \cite{wang2017deep}.
In this work, we introduce a novel deep learning based covert communication method inspired by steganography techniques using generative adversarial networks (GANs). The proposed covert scheme is designed for the next generation autoencoder-based wireless communication models, which are believed to be replacing traditional modular based wireless systems in the near future. However, there is no limitation on integrating our covert model to the current wireless communication systems.\\
The contributions of this work can be expressed as:
\begin{enumerate}
	\item We propose a novel covert communication method using GANs that works independent of the channel and cover signals of the wireless system.
	\item We train and evaluate our system on two different channel models of AWGN and Rayleigh fading to be more compliant to a realistic wireless communication model.
	\item We also measured both the performance of our covert model and the normal autoencoder-based communication system in terms of error rates to comprehend effectiveness of covert scheme and the impact of it on the existing communication between normal users.
\end{enumerate}
