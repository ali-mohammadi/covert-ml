\section{Introduction}
\label{s:intro}
Due to the broadcast nature of wireless channels, there is considerable attention to the security and privacy aspects of wireless communications. While the content of messages (i.e the information transmitted over the channel) can be protected against unauthorized access using cryptography or physical-layer security techniques \cite{zhou2013physical}, there are occasions when hiding the very existence of the communication channel is as vital as securing the communicated messages themselves. Examples of such situations are military operations, cyber-espionage, social unrest, or communication parties' privacy. All above have motivated the study of hidden communication channels, which are referred to as ``covert communication'' \cite{lampson1973note} in the literature.

There is a large body of work focusing on different schemes of covert communication applied into traditional wireless systems, but wireless systems have been going through tremendous changes during the past few years. This is mostly owing to the rise of machine learning in a variety of fields, including wireless communication \cite{wang2017deep}. That being so, it requires us to study the applicability of covert communication models in the next generation of wireless communication systems that are mostly machine learning driven. Specifically, various network optimization problems, which were traditionally used to be handled with statistical models, are now leveraging machine learning techniques. Deep neural networks (DNNs) in particular, the major force in machine learning, have answered several wireless problems, such as signal classification, channel estimation, transmitter identification, jamming and anti-jamming \cite{bahramali2021robust}. Very recently, as a replacement for conventional modular-based designs, an end-to-end communication model based on deep learning methods has emerged \cite{o2017introduction}. In this new paradigm, the transmitter and receiver are jointly trained as the encoder and decoder of an autoencoder network \cite{baldi2012autoencoders}. The autoencoder network can be trained to learn the characteristics of the signal, such as its statistical properties, and can then be used to perform compression or denoising on the signal. This improves the efficiency of the wireless transmission and helps to reduce the amount of errors that occur during the transmission. One noticeable difference between end-to-end systems and conventional modular designs is that in end-to-end systems, the encoder and decoder learn the coding and modulation tasks simultaneously instead of having separate modules for each. This gives autoencoder networks an advantage to enhance the error-correction capability of wireless systems by identifying and rectifying errors that happen during the transmission.

The preliminary attempt to obtain covertness started with the study of spread spectrum almost a century ago with the main purpose of hiding military communications \cite{scholtz1982origins}. The idea was to transmit the signal over a wide frequency band, which would make it harder to locate and identify the original signal amidst the background noise. Many works continued to further examine different aspects of this idea, however, the fundamental performance limits of such work were unknown until recently when Bash et al. \cite{bash2012square} established a square root limit on the number of covert bits that can be reliably sent over an additive white Gaussian noise (AWGN) channel while staying covert to a channel's observer. Followed by this work, there has been a surge of interest in examining covert channels \cite{sobers2017covert,soltani2018covert}, especially in point-to-point wireless communication systems.


Numerous works have studied the theoretical limits of covert communication over wireless systems in different scenarios \cite{bash2012square, soltani2018covert, sheikholeslami2018multi, li2021fundamental}, but only a few works have focused on practical implementation of covert communication. In a typical covert communication scenario, there is a specific element of the communication system that covert users leverage to build their covert communication upon. Some of which to mention are hardware impairments \cite{mohammed2021adversarial}, the channel's noise effect \cite{soltani2018covert}, presence of a cooperative jammer \cite{sobers2017covert}. Beyond that, the majority of works make some favorable assumptions for covert users, such as existing a shared secret key between Alice and Bob unknown to Willie \cite{soltani2018covert}, accessibility of covert users to cover signals and modulation type \cite{grzesiak2021wireless}, uncertainty in the knowledge of noise power at Willie's receiver \cite{he2017covert}, neglecting the impact of the covert system on the normal communication \cite{mohammed2021adversarial}, and limiting the channel model to AWGN \cite{mohammed2021adversarial}. Imposing such restricted assumptions eliminates the generality of these covert models and makes them difficult to adapt to systems with distinct conditions in different deployments. That said, these works have successfully demonstrated novel and unique methods of covert communication, but it is always encouraged to pursue new methods that are versatile and free of many of these assumptions.


In this work, we introduce a novel deep learning-based covert communication method that can be augmented to any autoencoder-based wireless communication and establish a reliable covert channel on top of the existing normal communication. To the best of our knowledge, this is the first covert scheme that is proposed for the recently introduced autoencoder wireless systems. There are only a few works in the literature that exploit machine learning techniques to develop a covert communication. However, there are some shortcomings in those works that ours tries to address. Specifically, in the covert scheme proposed by Liao et al. \cite{liao2020generative}, there is an assumption that the covert power allocator network has access to multiple environmental variables such as the channel matrix, noise power, and the transmission rates. Apart from that, they model relies on an external relay node that cooperates with covert users. In \cite{mohammed2021adversarial}, Mohammed et al. introduce a covert model that covert signals are added to the system in the form of hardware impairments. However, the impact of these added signals to the normal communication of the system is unknown. Besides, the only channel model considered is AWGN, which does not accurately reflect channel effects in real-world wireless communications. Our proposed covert communication model relies on no external factor but the existing channel's noise and is verified to work independently of the channel model. Moreover, it requires no knowledge of cover signals, the modulation type, or the number of users in the system and is optimized to have the minimum impact on the normal communication while being stealth to a watchful system observer. It is worthwhile to mention that even though we are proposing our method for autoencoder-based wireless communication systems, there is no limitation on integrating our model into existing conventional wireless communication systems.


The contributions of this work can be expressed as follows:
\begin{enumerate}
	\item \textbf{Cover-Agnostic}: We propose a novel covert communication method using GANs that is independent of cover signals, waveforms, and modulation types of wireless systems.
	\item \textbf{Channel Independent}: In our proposed covert scheme, we train our model on three different channel models of AWGN, Rayleigh Fading, and Rician Fading and and show our model does not rely on any information about the channel model or the noise level.
	\item \textbf{Single-/Multi-User Adaptability}: We show that our work can be integrated both into single-user and multi-user communication setups and in either case it works reliably. We also demonstrate there is a degree of freedom effect in our scheme, where increasing the number of users affects the performance of the covert and normal communication of the system.
	\item \textbf{Undetectability}: By forming an adversarial competition between observer and covert users wherein one competes to outperform the other, we find an optimal solution that ensures observer of the system will be close to random guessing in differentiating the covert and non-covert transmissions, despite being trained on labeled covert and normal signals during training.
	\item \textbf{Low Interference}: By evaluating the normal communication BLER after applying our covert model we show that our GAN-based covert model can be integrated into any normal communication system with having almost no disturbance on the ongoing normal communication.
\end{enumerate}

Section ? presents
