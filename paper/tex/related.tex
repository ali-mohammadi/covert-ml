\section{Related Works}
\label{s:related}
The main idea of our work stems from stenography techniques, which are commonly used in image steganography. Hence, we first briefly talk about the history and current state of this field of research and then we continue reviewing some of the existing approaches of establishing covert communication at physical layer in a wireless network.


\textbf{Image Steganography}: Deep learning algorithms have proved their efficiency in many aspects. Steganography is one of these areas that has benefited tremendously from deep learning advancements in recent years. The earliest use cases were in steganalysis research. Convolution neural networks (CNNs) for instance, which are generally used in computer vision tasks, showed outstanding results in image steganalysis \cite{tan2014stacked,qian2015deep,xu2016structural}. One of the earliest works on image steganography using deep neural networks is a work by \cite{baluja2017hiding}. In this work, Baluja proposes a hiding scheme in which the three networks of preparation, hiding, and reveal sort out the secret encoding and decoding task. The disadvantage of these schemes, however, was that the encoding process is reliant on the cover image. To address this, Zhang et al. \cite{zhang2020udh} propose a new architecture in which secret message can be encoded independent of cover images. To manifest robustness against steganalysis practices, researchers started to adopt GAN architectures \cite{goodfellow2014generative}. Hayes et al. \cite{hayes2017generating} introduce a GAN-based steganography technique that has a different objective for the generator network. Instead of generating cover images, the generator directly learns to embed secret messages into cover images so that the discriminator cannot find the differences. Although this adversarial scenario was preliminary introduced for hiding data in images, researchers found it so versatile that it has now been applied into other forms of data such as video, audio and recently wireless signals.


\textbf{Covert Communication}: One of the first attempts to implement a real-world covert communication is the work by Dutta et al. \cite{dutta2012secret}. They leverage the communication noise caused by either the channel or the hardware imperfections to establish a covert channel. In their proposed method, messages are covertly encoded in the constellation error of normal cover signals. Similarly, Cao et al. \cite{cao2018wireless} further improve this method with the goal of reducing the probability of detection. Hou et al. \cite{hou2020cloaklora} propose an amplitude based covert channel over LoRa PHY. In their scheme, covert information is embedded with a modulation scheme orthogonal to chirp spread spectrum (CSS). The disadvantage of all the above methods is that they cause distortions in the statistical properties of the system, thus can be detected by an observer. More recent works have explored the viability of deep neural networks in covert communication problem. Sankhe et al. \cite{sankhe2019impairment} propose a method called Impairment Shift Keying that produces subtle variations in normal signals that can be decoded by CNNs. To achieve the highest covert rate while minimizing the probability of detection, Liao et al. \cite{liao2020generative} employ a GAN model that can adaptively adjust the signal power at the covert sender. Motivated by the GAN-based steganography technique, Mohammed et al. \cite{mohammed2021adversarial} formulate the covert communication as a three-player game in which three networks are jointly trained. In this setup, the encoder and decoder networks learn to covertly communicate through a form of noise and simultaneously try to confuse a detector network that is responsible for differentiating the expected noise of the system from the added covert perturbations. 


While our proposed method shares the same idea as of aforementioned work, there are a couple of deficiencies that our work aims to address. First, our model is not leveraging on any hardware impairment noise and embeds the covert signals into the existing channel's noise. Second, our covert model is completely independent of cover signal whereas covert users of previous work are dependent on the knowledge of cover signals. Third, in the previous work, the impact of added covert signals to the communication in unknown, whereas our model is optimized to preserve the performance of normal communication. Finally yet importantly, previous work assumes the channel between users to be AWGN to simplify the problem, however, in a real wireless communication system, AWGN is not an accurate model to simulate the channel effects since signals are also subject to fading. Our work addresses this by considering two other channel models of Rayleigh and Rician fading.