\section{Related Works}
\label{s:related}
Since the main idea of our work stems from the stenography techniques, we first briefly go over the history and the current state of this field of research. We then continue this section reviewing some of the existing approaches of establishing covert communication at the physical layer of wireless networks.


\textbf{Image Steganography}: Deep learning algorithms have proven their efficiency in many aspects. Steganography is one of these areas that has benefited tremendously from deep learning advancements in recent years. Convolution neural networks (CNNs) for instance, which are generally used in computer vision tasks, have shown outstanding results in image steganalysis \cite{tan2014stacked,qian2015deep,xu2016structural}, replacing the traditional statistical methods. One of the earliest works of image steganography using deep neural networks is a work by \cite{baluja2017hiding}. In this work, Baluja proposes a hiding scheme in which the three networks of preparation, hiding, and reveal sort out the secret encoding and decoding task. The preparation network transforms the hidden message into features that are commonly used for compressing images. Then, the hiding network embeds it into the cover image and sends it to the reveal network, where the secret message gets extracted from the container image. Following this work, it was discovered that the existence of preparation network is not necessary and the framework can be expressed in a simpler form by excluding this network \cite{zhang2021brief}. The disadvantage of these schemes is that the encoding process is reliant on the cover image. To address this, Zhang et al. \cite{zhang2020udh} propose a new architecture in which secret message can be encoded independent of the cover image. Beside having more flexibility on hiding the information, this approach has also become an effective method for image watermarking. To manifest robustness against steganalysis practices, researchers started to adopt generative adversarial network (GAN) architectures \cite{goodfellow2014generative}. Volkhonskiy et al. \cite{volkhonskiy2020steganographic} propose one of the first steganography techniques based on GAN networks. The main idea of their work is to use a generative network to produce a new set of cover images that when carries the secret message using any of the available steganography techniques will be less exposed to be detected by a discriminator network (i.e. a steganalysis network). Similarly, Hayes et al. \cite{hayes2017generating} introduce a GAN-based steganography technique with a different objective for the generator network. Instead of generating cover images, the generator learns to embed secret messages into the cover images so that the discriminator cannot distinguish cover images from stenographic images. Although this adversarial scenario was preliminary introduced for hiding data in images, researchers found it so versatile that it has been applied into the other forms of data such as video, audio and text. This has inspired us to investigate the applicability of such techniques in wireless signals domain.


\textbf{Covert Communication}: Covert communication has been applied to almost every network layer and protocol, such as IP \cite{cabuk2004ip}, MAC \cite{sheikholeslami2020covert}, and DNS \cite{nussbaum2009robust}. However, covert communication at the physical layer has not been receiving much attention until recently. Dutta et al. \cite{dutta2012secret} are on of the first researchers who developed a covert communication technique for the physical layer of the wireless communication systems. They leverage the communication noise, which can be caused either by the channel or the hardware imperfections, as a mean to establish the covert channel. In their method, messages are covertly encoded in the constellation error of the normal cover signals. Similarly, Cao et al. \cite{cao2018wireless} further improved this method with the goal of reducing probability of detection. However, distortions that these methods cause in the statistical properties of the system were later found not so difficult to be detected using steganalysis methods \cite{huang2020exploiting}, which in turn results in compromising the channel. More recent works have explored the viability of deep neural networks in covert communication problem. Sankhe et al. \cite{sankhe2019impairment} propose a method called Impairment Shift Keying that produces subtle variations in normal signals in a controlled way such that a CNN model can be trained to classify them as zeros or ones. Although the impact of their covert method on the system's communication error rate is as small as 1\%,  authors in \cite{huang2021detection} showed that even tiny modifications to the signals' constellation points can be detected using a CNN model trained on the amplitude and phase characteristics of the error vector magnitude (EVM) and constellation points. Besides, their scheme relies on existing hardware impairments in the system, which is not the case in many deployments. To find an optimal solution for the highest covert rate and minimum probability of detection, Liao et al. \cite{liao2020generative} employed a GAN model that can adaptively adjust the signal power at a relay station for establishing a covert communication. This requires the covert users to have access to a cooperative relay node that is not the case in many communication scenarios. Another example of adversarial training for covert communication can be found in \cite{kim2022covert}. Their setup contains a transmitter communicating with a receiver through a Reconfigurable Intelligent Surface (RIS), and their goal is to keep this communication covert from a prospective eavesdropper. Both the intended receiver and the eavesdropper use CNN classifiers to detect the signals. This scenario raises the same concern of the previous work's that is the necessity of existence of a relay node in the deployment. Moreover, perturbations added to the signals to deceive the eavesdropper are crafted using fast gradient method (FGM) \cite{goodfellow2014explaining}, which in \cite{bahramali2021robust} was shown easy to counter using the existing countermeasure techniques. Our work differs from these two works in two aspects. First, our proposed method relies on no external entities or external factors, such as relay nodes and hardware impairments, and this helps with the generality of our model. Second, we use a generative network to produce a set of perturbations to maximize the robustness of our attacks against the system's observer or eavesdropper. 

One of the most related works to ours is the covert scheme proposed by Mohammed et al. \cite{mohammed2021adversarial}. They formulate the covert communication as a three-player game in which networks compete in an adversarial setting to obtain the optimal solution. In their setup, the encoder and decoder networks learn to covertly communicate through a form of noise and simultaneously try to confuse a detector network that tries to tell whether the users are covertly communicating or it is just the normal transmissions going through. While our proposed method shares some of the ideas of aforementioned work, there are a couple of limitations in this work that ours aims to address. First, our model is not leveraging on any hardware impairment noise and embeds the covert signals into the existing channel's noise. Second, in the previous work, the impact of added covert signals on communication is unknown, but our model is optimized to preserve the performance of normal communication. And finally, previous work assumes the channel between users to be AWGN, however, this is not an accurate model for simulating the channel effects in wireless communications due to fading. Our work addresses this by proposing a robust method against different channel models including Rayleigh and Rician fading channels.