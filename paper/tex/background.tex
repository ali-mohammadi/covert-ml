\section{Background On Autoencoder Communication Systems}
\label{s:design}
An autoencoder-based communication system is an end-to-end learning paradigm that abstracts out the coding and modulation components of a  traditional modular communication system by replacing the transmitter and receiver with DNNs. You can see a block diagram of such a system in figure ?. The encoder (transmitter) first uses a mapping to transform \(k\) bits of data into a message \(s\) where \(s \in \{1,...,M\}\) and \(M = 2^k\). Then it takes this transformed message as an input and generates a signal \(x = E(s) \in \mathbb{R}^{2n}\), which is a real valued vector. This \(2 \times n\) dimensional real valued vector can be treated as an \(n\) dimensional complex vector where \(n\) is the number of channel uses the signal needs to be transmitted over. Then, the channel effect \(z\), which is usually considered to be AWGN, is added to the signal vector. Thus, the received signal at the receiver carries within iselfs noise of the channel and can be expressed as \(y = x + z\). The decoder (receiver) applies the transformation \(g: \mathbb{R}^{2n} \rightarrow M \) to outputs the reconstructed version of the message \(s\), which is denoted as \(\hat{s} = g(y)\).