\section{Related Works}
\label{s:related}
The main idea of our work stems from stenography techniques, which are commonly used in image steganography. Hence, we first briefly talk about the history and current state of this field of research and then we continue reviewing some of the existing approaches of establishing covert communication at physical layer in a wireless network.\\
Deep learning algorithms have proved their efficiency in many aspects. Steganography is one of these areas that has benefited tremendously from deep learning advancements in recent years. The earliest use cases were in steganalysis research. Convolution neural networks (CNNs) for instance, which are generally used in computer vision tasks, showed outstanding results in image steganalysis \cite{tan2014stacked,qian2015deep,xu2016structural}, replacing the traditional statistical methods. One of the earliest works on image steganography using deep neural networks is a work by \cite{baluja2017hiding}. In this work, Baluja proposes a hiding scheme in which the three networks of preparation, hiding, and reveal sort out the secret encoding and decoding task. The preparation network transforms the hidden message into features that are commonly used for compressing images. Then, the hiding network embeds it into a cover image and sends it to the reveal network, where the secret message gets extracted from the cover image. Followed by this work, researchers discovered that the existence of preparation network is not necessary and the framework can be expressed in a simpler form by excluding this network from the model \cite{zhang2021brief}. The disadvantage of these schemes, however, was that the encoding process is reliant on the cover image. To address this, Zhang et al. \cite{zhang2020udh} propose a new architecture in which secret message can be encoded independent of cover images. Beside having more flexibility on hiding the information, this approach has also become an effective method for image watermarking. To manifest robustness against steganalysis practices, researchers started to adopt GAN architectures. In a typical adversarial network setting there are two neural networks involved, a generator network (G) and a discriminator network (D). These two networks are then trained against each other where the generator tries to deceive the discriminator by generating data similar to those of in the training set and the discriminator tries to correctly categorize them as fake and real \cite{goodfellow2014generative}. Volkhonskiy et al. \cite{volkhonskiy2020steganographic} propose a steganography technique based on GAN training. The main idea of their work is to use a generative network to produce a new set of cover images that when carry the secret message using any of the available steganography techniques are less exposed to be detected by a discriminator network (i.e. a steganalysis network). Similarly, Hayes et al. \cite{hayes2017generating} introduce a GAN-based steganography technique that has a different objective for the generator network. Instead of generating cover images, the generator directly learns to embed secret messages into cover images so that the discriminator cannot find the differences. Although this adversarial scenario was preliminary introduced for hiding data in images, researchers found it so versatile that it has now been applied into other forms of data such as video, audio and recently wireless signals.\\
Numerous works have studied the theoretical limits of covert communication over wireless channels in different scenarios \cite{bash2012square, soltani2018covert, sheikholeslami2018multi, li2021fundamental}, but only few works have focused on a practical implementation of such channels. One real world example of a covert communication is the work by Dutta et al. \cite{dutta2012secret}. They leverage the communication noise caused by either the channel or the hardware imperfections to establish a covert channel. In their proposed method, messages are covertly encoded in the constellation error of normal cover signals. Similarly, Cao et al. \cite{cao2018wireless} further improve this method with the goal of reducing the probability of detection. Hou et al. \cite{hou2020cloaklora} propose an amplitude based covert channel over LoRa PHY. In their scheme, covert information is embedded with a modulation scheme orthogonal to chirp spread spectrum (CSS). Bonati et al. \cite{bonati2021stealte} introduce SteaLTE which is a full-stack wireless steganography method on softwarized cellular networks. To covertly modulate symbols, they employ the three different approaches of dirty QPSK modulation, hierarchical amplitude shift keying (ASK) manipulation, and phase offset of the primary symbols modification. The disadvantage of all the above methods is that the distortions caused in the statistical properties of the system can be detected with high confidence by a careful observer. More recent works have explored the viability of deep neural networks in covert communication problem. Sankhe et al. \cite{sankhe2019impairment} propose a method called Impairment Shift Keying that produces subtle variations in normal signals in a controlled way such that a CNN model can be trained to classify them as zeros or ones. To achieve the highest covert rate while minimizing the probability of detection, Liao et al. \cite{liao2020generative} employ a GAN model that can adaptively adjust the signal power at the covert sender. Motivated by the GAN-based steganography technique, Mohammed et al. \cite{mohammed2021adversarial} formulate the covert communication as a three-player game in which three networks are jointly trained. In this setup, the encoder and decoder networks learn to covertly communicate through a form of noise and simultaneously try to confuse a detector network that is responsible for differentiating the expected noise of the system from the added covert perturbations. While our proposed method shares the same idea as of this work, there are a couple of deficiencies in the previous work that ours aims to address. First, we study covert communication over Autoencoder-based wireless systems which will soon replace the traditional modular systems in future wireless communication technologies such as 6G. Second, the authors of previous work state that their covert scheme is independent of cover signals, however, they assume the modulation type is known to the covert receiver; thus, the covert receiver knows what the cover signal is and only by subtracting it from the received signal is that the receiver can recover the covert message. In our work, however, covert receiver extracts the secret message independent of this knowledge. Third, the previous work evaluates the performance of their proposed covert communication model but gives no information on the impact of their added noise on the normal communication of the system. It is of utmost importance that the added covert noise signals do not interfere with the normal communication of the system, since any unexpected increase in the communication's error rate would be an indication of a suspicious activity (i.e. in this case, a covert communication) taking place. Finally yet most importantly, previous work assumes the channel between users to be AWGN to simplify the problem, however, in a real wireless communication system, AWGN is not an accurate model to simulate the channel effects since signals are also subject to fading. Our work addresses this by also considering Rayleigh fading as the communication channel.