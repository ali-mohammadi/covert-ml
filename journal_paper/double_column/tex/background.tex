\section{Autoencoder Wireless Systems}
\label{s:back}

Autoencoder-based communication systems are a relatively new development in wireless communications and have numerous advantages, such as their ability to learn from data and adapt to changing conditions in the wireless environment, making them ideal for dynamic wireless environments. They can also be trained to handle noise and errors in the transmitted data, not only for linear, but also for nonlinear channel effects. Autoencoder-based wireless communication systems were introduced as an end-to-end learning paradigm that abstracts the coding and modulation components of traditional modular communication systems by replacing the transmitter and receiver with DNNs. The encoder (transmitter) learns the underlying statistical properties of the wireless channel and accordingly modulates the signal, while the decoder (receiver) receives a noisy version of the signal and attempts to reconstruct the message. As shown in Fig. \ref{fig:original_autoencoder_architecture}, these systems use autoencoder neural network designs \cite{baldi2012autoencoders} to enhance the performance of wireless communication systems. We describe below how this communication works in both single-user and multi-user systems in greater detail.

\textbf{Single-User Autonencoder Systems}: In this type of communication system, the encoder transforms \(k\) bits of data into a message \(s\) where \(s \in \{1,...,M\}\) and \(M = 2^k\). The encoder then takes this transformed message as an input and generates a signal \(X = E(s) \in \mathbb{R}^{2n}\), which is a real-valued vector. This \(2 \times n\)-dimensional real-valued vector can be treated as an \(n\)-dimensional complex vector, where \(n\) is the number of channel uses required for signal transmission. To account for channel noise, usually additive white Gaussian noise (AWGN), the noise effect is added to the signal vector, denoted as \(N\), which is a \(n\)-dimensional independent and identically distributed (i.i.d.) vector with each element coming from a complex normal distribution with 0 mean and \(\sigma^2\) variance \(N_i \sim \mathcal{CN}(0, \sigma^2)\). A fading coefficient \(H\) is introduced to account for the varying channel conditions. If there is no fading, \(H\) is equal to the identity matrix \(I_n\). However, in the presence of fading, \(H\) is a diagonal matrix with each element following a fading distribution, such as Rayleigh or Rician. Ultimately, the received signal at the receiver, carrying the channel's noise, can be expressed as \(Y = H \cdot X + N\). Once the signals pass through the channel, the decoder receives these distorted signals and applies the transformation \(D: \mathbb{R}^{2n} \rightarrow M \) to output the reconstructed version of the message \(s\), which is  denoted as \(\hat{s} = D(Y)\).

\textbf{Multi-User Autonencoder Systems}: A multi-user autoencoder communication system is an extended version of a single-user system with either multiple transmitters and receivers or multiple transmitters and a central receiver. In this work, we consider the latter case. In this system, each encoder sends a separate message \(s_i\) where \(s_i \in s\), and \(i\) is the index of the transmitter. Each transmitter generates a modulated signal accordingly: \(X_i = E_i(s_i) \in \mathbb{R}^{2n}\). We consider a multiple-access system, where all transmitters send their signals at the same time. Consequently, the signals experience channel interference in addition to the channel effects. When passing through the channel, signals are simultaneously faded and interfere with each other. The resulting vector of signals for each transmitter can be expressed as \(Y_i = \sum_{i=1}^{n_{tx}} H_i \cdot X_i \cdot e^{j\theta_i} + N_i\), where \(H_i\) is the channel coefficient and \(e^{j\theta_i}\) is the phase offset for the \(i^{th}\) transmitter, \(X_i\) is the corresponding encoded signal, and \(n_{tx}\) is the number of transmitters. Finally, the signals are received at the decoder, where it uses its decoding function \(D(\cdot)\) along with the channel matrix \(H\) with the size of \(n_{tx} \times n_{rx}\), where \(n_{rx}\) is the number of antennas, to reconstructed the message \(\hat{s} = D(Y, H)\).