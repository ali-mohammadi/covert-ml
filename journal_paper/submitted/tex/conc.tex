\section{Conclusion}
\label{s:conc}
In this paper, we present a novel deep learning-based covert communication approach that embeds a secret message into a covert signal without the need for handcrafted features. By employing the generative adversarial training framework, we significantly reduce the detection probability of the produced covert signals. Additionally, our proposed training procedure enables us to adjust the trade-off between covert communication reliability and probability of detection by incorporating regularizers in the model's training loss function, regardless of the channel conditions or user messages. Our results show that our covert model is channel-agnostic and insensitive to cover signals. We evaluate the performance of our model across three channel models, namely AWGN, Rayleigh, and Rician fading, and for various covert rates and the number of system users. Furthermore, we investigate the impact of our added covert signals on the ongoing normal system and demonstrate minimal disruption caused by our covert scheme.

Future research could explore the applicability of our covert scheme in blackbox systems, where covert users have no access to the normal users' communication networks. This would involve training our scheme against a substitute normal receiver and evaluating its effectiveness in such scenarios.