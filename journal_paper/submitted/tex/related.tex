\section{Related Works}
\label{s:related}
Since the main idea of our work stems from steganography techniques, we first briefly go over the history and current state of this field of research. We then continue this section by reviewing some of the existing approaches to establishing covert communication at the physical layer of wireless networks.

\textbf{Image Steganography}: Deep learning algorithms have greatly advanced various fields, including steganography. Convolutional neural networks (CNNs), originally used in computer vision tasks, have demonstrated remarkable performance in image steganalysis, surpassing traditional statistical methods \cite{tan2014stacked, qian2015deep, xu2016structural}. One of the earliest works of image steganography using deep neural networks is by Baluja \cite{baluja2017hiding}. In this work, Baluja proposes a hiding scheme in which the three networks of preparation, hiding, and reveal sort out the secret encoding and decoding task. The preparation network transforms the hidden message into compressed image features. The hiding network then embeds these features into the cover image, which is processed by the reveal network to extract the secret message. Subsequent research has found that the preparation network can be omitted, simplifying the framework without compromising its effectiveness \cite{zhang2021brief}. The disadvantage of these schemes is that the encoding process is reliant on the cover image. To address this, Zhang et al. \cite{zhang2020udh} propose a new architecture in which the secret message can be encoded independently of the cover image. Besides having more flexibility in hiding the information, this approach has also become an effective method for image watermarking. To manifest robustness against steganalysis practices, researchers started to adopt GAN architectures \cite{goodfellow2014generative}. Volkhonskiy et al. \cite{volkhonskiy2020steganographic} propose one of the first steganography techniques based on GAN networks. The main idea of their work is to use a generative network to produce a new set of cover images that, when carrying the secret message using any of the available steganography techniques, will be less exposed to be detected by a discriminator network (i.e., a steganalysis network). Similarly, Hayes et al. \cite{hayes2017generating} introduce a GAN-based steganography technique with a different objective for the generator network. Instead of generating cover images, the generator learns to embed secret messages into the cover images so that the discriminator cannot distinguish cover images from steganographic images. Although this adversarial scenario was initially introduced for hiding data in images, researchers found it so versatile that it has been applied to other forms of data such as video, audio, and text \cite{martin2023evolving}. This has inspired us to investigate the applicability of such techniques in the wireless communication domain.


\textbf{Covert Communication}: Covert communication has been applied to almost every network layer and protocol, such as IP \cite{cabuk2004ip}, MAC \cite{sheikholeslami2020covert}, and DNS \cite{nussbaum2009robust}. However, covert communication at the physical layer has received little attention until recently. Dutta et al. \cite{dutta2012secret} are one of the first researchers who developed a covert communication technique for the physical layer of wireless communication systems. They leverage communication noise, which can be caused by either the channel or hardware imperfections, as a means to establish the covert channel. In their method, messages are covertly encoded in the constellation error of the normal cover signals. Similarly, Cao et al. \cite{cao2018wireless} further improved this method with the goal of reducing the probability of detection. However, the distortions that these methods cause in the statistical properties of the system were later found not to be difficult to detect using steganalysis methods \cite{huang2020exploiting}, which in turn compromises the covert channel. More recent works have explored the viability of deep neural networks in the covert communication problem. Sankhe et al. \cite{sankhe2019impairment} propose a method called Impairment Shift Keying that produces subtle variations in normal signals in a controlled way such that a CNN model can be trained to classify them as zeros or ones. Although the impact of their covert method on the system's communication error rate is as small as 1\%, authors in \cite{huang2021detection} showed that even tiny modifications to the signals' constellation points can be detected using a CNN model trained on the amplitude and phase characteristics of the error vector magnitude (EVM) and constellation points. Besides, their scheme relies on existing hardware impairments in the system, which is not the case in many deployments. To find an optimal solution for the highest covert rate and minimum probability of detection, Liao et al. \cite{liao2020generative} employed a GAN model that can adaptively adjust the signal power at a relay station for establishing covert communication. This requires the covert users to have access to a cooperative relay node, which is not the case in many communication scenarios. Another example of adversarial training for covert communication can be found in \cite{kim2022covert}. Their setup contains a transmitter communicating with a receiver through a Reconfigurable Intelligent Surface (RIS), and their goal is to keep this communication covert from a prospective eavesdropper. Both the intended receiver and the eavesdropper use CNN classifiers to detect the signals. This scenario raises the same concern as the previous work, which is the necessity of the existence of a relay node in the deployment. Moreover, perturbations added to the signals to deceive the eavesdropper are crafted using the fast gradient method (FGM) \cite{goodfellow2014explaining}, which in \cite{bahramali2021robust} was shown to be easy to counter using existing countermeasure techniques. Our work differs from these two works in two key aspects. First, our proposed method does not rely on any external entities or external factors, such as relay nodes or hardware impairments, which helps with the generality of our model. Second, we have specifically designed our covert model to have as little impact on normal communication as possible, ensuring that the error rate of normal communication does not suspiciously rise. This objective is critical and often overlooked in previous works, which could easily expose any covert communication to the system's observer.

One of the most related works to ours is the covert scheme proposed by Mohammed et al. \cite{mohammed2021adversarial}, where covert communication is formulated as a three-player adversarial game. In their setup, the encoder and decoder networks learn to covertly communicate through noise, while a detector network aims to differentiate covert communication from normal transmissions. While our method shares some ideas with this work, it addresses several critical limitations. First, our model embeds covert signals into the existing channel's noise instead of relying on hardware impairment noise. Second, we optimize our model to preserve the performance of normal communication. Lastly, in contrast to the previous work, which assumes only AWGN channel model, our approach is robust and considers fading channel models like Rayleigh and Rician fading channels, in addition to AWGN.