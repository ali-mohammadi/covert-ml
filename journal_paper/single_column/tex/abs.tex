	
\begin{abstract}
	Hiding the wireless communication by transmitter Alice to intended receiver Bob from a capable and attentive adversary Willie has been widely studied under the moniker ``covert communications''.  However, when such covert communication is done in the presence of allowable system communications, there has been little study of both hiding the signal and preserving the performance of those allowable communications. Here, by treating Alice, Bob, and Willie as a generator, decoder, and discriminator neural network, we perform joint training in an adversarial setting to yield a covert communication scheme that can be added to any normal autoencoder. The method does not depend on the characteristics of the cover signal or the type of channel and it is developed for both single-user and multi-user systems. Numerical results indicate that we are able to establish a reliable undetectable channel between Alice and Bob, regardless of the cover signal or type of fading, and that the signal causes almost no disturbance to the ongoing normal operation of the system.
\end{abstract}

% Note that keywords are not normally used for peerreview papers.
\begin{IEEEkeywords}
	Wireless security, covert communication, autoencoder systems, adversarial training.
\end{IEEEkeywords}


	
% For peer review papers, you can put extra information on the cover
% page as needed:
% \ifCLASSOPTIONpeerreview
% \begin{center} \bfseries EDICS Category: 3-BBND \end{center}
% \fi
%
% For peerreview papers, this IEEEtran command inserts a page break and
% creates the second title. It will be ignored for other modes.
\IEEEpeerreviewmaketitle
